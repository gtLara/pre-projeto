\documentclass{article}

\usepackage[utf8]{inputenc}
\usepackage[portuguese]{babel}


\begin{document}

\textbf{\large Análise e Processamento de Séries Temporais com Aplicações em Investimento Algorítmico e Finanças Quantitativas}

O pré projeto de pesquisa deve descrever as atividades de pesquisa em computação
que o candidato pretende desenvolver, considerando a abrangência temático do
PPGCC e as áreas de concentração em pesquisa de seu corpo docente, disponíveis
no site do Programa.

\subsubsection*{Introdução}

Demonstrar capacidade contextualização do tema de pesquisa e objetivos,
aderência do tema às áreas de pesquisa do PPGCC, objetivos

\subsubsection*{Referencial Teórico}

Demonstrar conhecimento da linha de pesquisa escolhida destacando em que
pontos a proposta de projeto poderá contribuir na expansão do estado da arte

\subsubsection*{Metodologia}

Demonstrar clareza em dar soluções para a linha de pesquisa escolhida

\subsubsection*{Cronograma}

Demonstrar exequibilidade da proposta, indicar possíveis disciplinas a cursar
e a organização do tempo durante seu período de vínculo ao curso

Disciplinas:

Núcleo Algoritmos: Projeto e Análise de Algoritmos, Programação Competitiva
Núcleo Estatística: FECD A, FECD B, Análise de Séries Temporais
Núcleo Otimização: Programação Linear, Programação Não Linear
Núcleo Específico: Finanças Quantitativas e Gerenciamento de Risco

1 semestre: PAA, FECD B
2 semestre: Machine Learning, Programação Competitiva
3 e 4: Programação Linear, Programação Não Linear, Análise de Séries Temporais,
Finanças Quantitativas (sujeito à oferta)

\subsubsection*{Referências}

\end{document}
