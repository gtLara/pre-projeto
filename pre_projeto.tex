\documentclass[a4paper, 12pt]{article}

\usepackage[utf8]{inputenc}
\usepackage[portuguese]{babel}

\begin{document}

\textbf{Modelagem Multivariada de Heteroscedasticidade para estimativa de Risco
Dinâmico em Otimização de Portfólio}

\subsubsection*{Introdução}

O problema de otimização de portfólio é um desafio fundamental na teoria
financeira que envolve a alocação eficiente de recursos em diferentes ativos. O
objetivo é encontrar a combinação ideal de investimentos que maximize o retorno
esperado para um determinado nível de risco, levando em conta restrições de
investimento como limitações de alocação e liquidez. Central ao problema é a
definição de uma medida de risco, responsável por quantificar o potencial de
perda financeira associado à um portfólio, sendo este um tópico de considerável
atenção acadêmica~\cite{best_risk_measure, righi2018, hoe2010, ramos2023}.

A medida de risco tradicional em otimização de portfólio é variância, seguindo
o método de Média-Variância proposto por Markowitz~\cite{markowitz}. Apesar de
conveniente devido à sua simplicidade esta medida não possui muitas
propriedades desejáveis~\cite{rockafellar2002deviation}, tornando comum o uso
de medidas mais elaboradas~\cite{gambrah2014risk, sereda, adam2008spectral}. Um
dos problemas da variância, compartilhado por muitas medidas mais sofisticadas,
é a suposição que risco é estático no tempo, enquanto mercados financeiros são
dinâmicos e sujeitos à constante variação temporal~\cite{procacci}. Métricas de
risco que tentam capturar a natureza dinâmica de uma carteira de investimentos
são apropriadamente denominadas medidas de risco dinâmicas.

A incorporação de medidas de risco dinâmicas em otimização de portfólio pode
ser útil para a geração de representações mais precisas de risco
instantâneo~\cite{weirum} e para a construção de portfólios dinâmicos, cuja
distribuição de investimentos evolui ao longo do tempo por meio de políticas de
rebalanceamento~\cite{metin, holten}. Como várias medidas de risco são
calculadas a partir da matriz de covariância das séries temporais de preços ou
retornos dos ativos de uma carteira, a modelagem temporal da matriz de
covariância é uma forma de gerar métricas de risco dinâmicas que é bem
estudada na literatura ~\cite{zakamulin2015test, chan1999portfolio}.

A modelagem temporal da matriz de covariância de um conjunto de séries
temporais é essencialmente um problema de modelagem de heteroscedasticidade
multivariada. ~ mencionar os modelos de engle que são paramétricos, depois os
esforços de ML, depois os de DL. mencionar como que ML e DL pode resolver os
os problemas dos modelos tradicionais.

% O objetivo do projeto proposto é aprimorar soluções de otimização de portfólio
% a partir de uma medida de risco dinâmica obtida por meio de modelagem temporal
% multivariada de preços de instrumentos financeiros.

\subsubsection*{Referencial Teórico}

Demonstrar conhecimento da linha de pesquisa escolhida destacando em que
pontos a proposta de projeto poderá contribuir na expansão do estado da arte

\subsubsection*{Metodologia}

Demonstrar clareza em dar soluções para a linha de pesquisa escolhida

\subsubsection*{Cronograma}

Demonstrar exequibilidade da proposta, indicar possíveis disciplinas a cursar
e a organização do tempo durante seu período de vínculo ao curso

Disciplinas:

Núcleo Algoritmos: Projeto e Análise de Algoritmos, Programação Competitiva
Núcleo Estatística: FECD A, FECD B, Análise de Séries Temporais
Núcleo Otimização: Programação Linear, Programação Não Linear
Núcleo Específico: Finanças Quantitativas e Gerenciamento de Risco

1 semestre: PAA, FECD B
2 semestre: Machine Learning, Programação Competitiva
3 e 4: Programação Linear, Programação Não Linear, Análise de Séries Temporais,
Finanças Quantitativas (sujeito à oferta)

\bibliographystyle{ieeetr}
\bibliography{refs}

\end{document}
