\documentclass[a4paper, 12pt]{article}

\usepackage{titlesec}
\usepackage[utf8]{inputenc}
\usepackage[portuguese]{babel}

\begin{document}

\textbf{Modelagem Multivariada de Heteroscedasticidade para estimativa de Risco
Dinâmico em Otimização de Portfólio}

\subsubsection*{Introdução}

O problema de otimização de portfólio é um desafio fundamental na teoria
financeira que envolve a alocação eficiente de recursos em diferentes ativos. O
objetivo é encontrar a combinação ideal de investimentos que maximize o retorno
esperado para um determinado nível de risco, levando em conta restrições de
investimento como limitações de alocação e liquidez. Central ao problema é a
definição de uma medida de risco, responsável por quantificar o potencial de
perda financeira associado à um portfólio, sendo este um tópico de considerável
atenção acadêmica~\cite{best_risk_measure, righi2018, hoe2010, ramos2023}.

A medida de risco tradicional em otimização de portfólio é variância, seguindo
o método de Média-Variância proposto por Markowitz~\cite{markowitz}. Apesar de
conveniente devido à sua simplicidade esta medida não possui muitas
propriedades desejáveis~\cite{rockafellar2002deviation}, tornando comum o uso
de medidas mais elaboradas~\cite{gambrah2014risk, sereda, adam2008spectral}. Um
dos problemas da variância(separação sujeito verbo?), compartilhado por muitas
medidas mais sofisticadas, é a suposição que risco é estático no tempo,
enquanto mercados financeiros são dinâmicos e sujeitos à constante variação
temporal~\cite{procacci}. Métricas de risco que tentam capturar a natureza
dinâmica de uma carteira de investimentos são apropriadamente denominadas
``medidas de risco dinâmicas''.

A incorporação de medidas de risco dinâmicas em otimização de portfólio pode
ser útil para a geração de representações mais precisas de risco
instantâneo~\cite{weirum} e para a construção de portfólios dinâmicos, cuja
distribuição de investimentos evolui ao longo do tempo por meio de políticas de
rebalanceamento~\cite{metin, holten}. Como várias medidas de risco são
calculadas a partir da matriz de covariância das séries temporais de preços ou
retornos dos ativos de uma carteira, a modelagem temporal da matriz de
covariância é uma forma de gerar métricas de risco dinâmicas que é bem estudada
na literatura ~\cite{zakamulin2015test, chan1999portfolio}.

A modelagem temporal da matriz de covariância de um conjunto de séries
temporais é essencialmente um problema de modelagem de heteroscedasticidade
multivariada. Dentre os modelos tradicionais com esta finalidade, os propostos
por Engle~\cite{var_garch}~\cite{dcc} geram medidas de risco dinâmicas mais
representativas que suas contrapartidas estáticas~\cite{metin, holten, weirum}.
As dificuldades de especificação, identificação e complexidade computacional
dos modelos de Engle e seus derivados~\cite{silvennoinen2009multivariate},
resultantes principalmente de sua natureza paramétrica, sugerem o uso de
modelos de aprendizado de máquina como uma alternativa mais flexível.

A aplicação de algoritmos de aprendizado de máquina para modelagem temporal de
matrizes de covariância atinge resultados melhores que abordagens tradicionais
em alguns trabalhos~\cite{svr, ann1, ann2}. Aprendizado profundo, em
particular, apresenta resultados promissores em estudos recentes~\cite{dl1,
dl2, dl3?}.

O objetivo do projeto proposto é investigar soluções de otimização de portfólio
a partir de uma medida de risco dinâmica obtida por meio de modelagem
multivariada de heteroscedasticidade de preços e retornos de ativos
financeiros. Serão explorados modelos paramétricos, não paramétricos e híbridos
com uma ênfase em aprendizado profundo. (incluir isso?) Como objetivo
secundário será escrito um sistema para avaliação prática do método de
otimização de portfólio desenvolvido.

% É demonstrado na literatura que algoritmos de aprendizado de máquina não
% paramétricos são mais eficientes para alguns problemas de econometria
% financeira~\cite{hsu2016bridging}.


% O objetivo do projeto proposto é aprimorar soluções de otimização de portfólio
% a partir de uma medida de risco dinâmica obtida por meio de modelagem temporal
% multivariada de preços de instrumentos financeiros.

\subsubsection*{Referencial Teórico}

Demonstrar conhecimento da linha de pesquisa escolhida destacando em que
pontos a proposta de projeto poderá contribuir na expansão do estado da arte

\subsubsection*{Metodologia}

Demonstrar clareza em dar soluções para a linha de pesquisa escolhida

\subsubsection*{Cronograma}

Conforme estabelecido pela estrutura curricular do PPGCC será cursada a
disciplina Projeto e Análise de Algoritmos do Núcleo Comum. Das disciplinas das
linhas de pesquisa de Inteligência Artificial e Otimização há interesse do
candidato em cursar Fundamentos de Estatística para Ciência dos Dados B,
Aprendizado Profundo, Aprendizado de Máquina, Programação Não Linear e
Programação Estocástica. Das demais destaca-se a matéria Finanças Quantitativas
e Gerenciamento de Risco. As matérias serão cursadas de acordo com o
planejamento conjunto do aluno e orientador e oferta.

As atividades pertinentes à pesquisa do candidato serão distribuídas entre os
quatro semestres da seguinte maneira:
\begin{enumerate}
    \item Revisão de literatura e experimentos iniciais com base de dados M6.
    \item Início de desenvolvimento de sistema de otimização e elaboração do
    método proposto.
    \item Finalização do sistema de otimização. Experimentos com o método
    proposto. Ajustes no método.
    \item Avaliação de experimentos finais e redação de dissertação.
\end{enumerate}


\bibliographystyle{ieeetr}

\begingroup
\titleformat*{\section}{\fontsize{12pt}{12pt}\bfseries\selectfont}
% {\footnotesize
\bibliography{refs}%}
\endgroup

\end{document}
