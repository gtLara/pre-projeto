\documentclass[a4paper, 12pt]{article}

\usepackage[utf8]{inputenc}
\usepackage[portuguese]{babel}

\begin{document}

\textbf{\large Modelagem Multivariada de Risco Dinâmico para Otimização de
Portfólio}

\subsection*{Introdução}

O problema de otimização de portfólio é um desafio fundamental na teoria
financeira que envolve a alocação eficiente de recursos em diferentes ativos
financeiros. O objetivo é encontrar a combinação ideal de investimentos que
maximize o retorno esperado para um determinado nível de risco, levando em
conta restrições de investimento como limitações de alocação e liquidez.
Central ao problema é a definição de uma medida de risco representativa da
insegurança associada à um conjunto específico de ativos, ?sendo este um tópico
de considerável atenção acadêmica~\cite{seleda}.

A medida de risco mais comum em otimização de portfólio é variância, seguindo a
o método de Média-Variância proposto por Markowitz~\cite{markowitz}. Apesar de
conveniente devido à sua simplicidade esta medida não possui muitas
propriedades desejáveis~\cite{}, levando ao desenvolvimento da otimização a
partir de medidas mais completas~\cite{risk_measures_portopt} (mais completas
como? melhores como?). blabalUma notável falha da variância é a não incorporação
de informação temporal - isso seria daora etc - 

retomar papers antigos para não perder o fio



- Falar sobre método de Markowitz. Mencionar que risco é uma medida fundamental
e que é obtida por meio da variância total da matriz de covariâncias obtidas
de forma estática. Mencionar críticas existentes à questão dessa medida ser
estática.

O objetivo do projeto proposto é aprimorar soluções de otimização de portfólio
a partir de uma medida de risco dinâmica obtida por meio de modelagem temporal
multivariada de preços de instrumentos financeiros.

\subsubsection*{Referencial Teórico}

Demonstrar conhecimento da linha de pesquisa escolhida destacando em que
pontos a proposta de projeto poderá contribuir na expansão do estado da arte

\subsubsection*{Metodologia}

Demonstrar clareza em dar soluções para a linha de pesquisa escolhida

\subsubsection*{Cronograma}

Demonstrar exequibilidade da proposta, indicar possíveis disciplinas a cursar
e a organização do tempo durante seu período de vínculo ao curso

Disciplinas:

Núcleo Algoritmos: Projeto e Análise de Algoritmos, Programação Competitiva
Núcleo Estatística: FECD A, FECD B, Análise de Séries Temporais
Núcleo Otimização: Programação Linear, Programação Não Linear
Núcleo Específico: Finanças Quantitativas e Gerenciamento de Risco

1 semestre: PAA, FECD B
2 semestre: Machine Learning, Programação Competitiva
3 e 4: Programação Linear, Programação Não Linear, Análise de Séries Temporais,
Finanças Quantitativas (sujeito à oferta)

\bibliographystyle{ieeetr}
\bibliography{refs}

\end{document}
