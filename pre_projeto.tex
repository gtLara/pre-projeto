\documentclass[a4paper, 12pt]{article}

\usepackage[utf8]{inputenc}
\usepackage[portuguese]{babel}

\begin{document}

\textbf{\large Análise de Séries Temporais Multivariadas com Aplicações em
Investimento Algorítmico e Finanças Quantitativas}

O pré projeto de pesquisa deve descrever as atividades de pesquisa em computação
que o candidato pretende desenvolver, considerando a abrangência temático do
PPGCC e as áreas de concentração em pesquisa de seu corpo docente, disponíveis
no site do Programa.

\subsection*{Introdução}

Séries temporais podem ser definidas como sequências de observações realizadas
ao longo do tempo~\cite{chatfield}, representando uma ampla classe de dados
cuja análise é importante para inúmeras áreas como engenharia, economia,
biologia e meteorologia, para citar algumas. Um dos campos mais importantes da
análise de séries temporais é o estudo das associações entre múltiplas séries e o
estabelecimento de modelos dinâmicos multivariados para a representação dessas
relações conjuntas ao longo do tempo, denominado por análise de séries
temporais multivariadas~\cite{box}.

No domínio financeiro dados são frequentemente séries temporais, já que
existe uma evolução no tempo de preços de instrumentos financeiros, taxas de
juros e índices, sendo estes alguns exemplos. Observa-se o surgimento de
múltiplas séries financeiras sincronizadas, isto é, com amostragem idêntica, em
mercados de instrumentos como ações, \emph{commodities} e opções~\cite{tsay}.
Análise dessas séries temporais de forma multivariada é de interesse devido à
existência de fortes relações como cointegração, causalidade de Granger, e
diversas medidas de correlação~\cite{morettin}.

Desde o surgimento de sistemas de negociação eletrônica na década de 1970 o
componente algorítmico da tomada de decisões sobre investimentos a partir da
análise de séries temporais financeiras tem crescido, se tornando o maior fator
em alguns sistemas como os de \emph{High Frequency Trading}(HFT) ou uma
ferramenta essencial em sistemas de menor frequência como os de gerenciamento
de carteiras de investimentos~\cite{kissell}. Independente do nível de
dominância algorítmica o projeto de \emph{software} de sistemas financeiros
atuais é uma componente de alto impacto em seu desempenho.

Apesar de muitas técnicas de investimento algorítmico envolverem diretamente o
processamento de múltiplas séries temporais suas versões mais comuns nem sempre
usam de teoria multivariada para tomada de decisões. Na literatura argumenta-se
que a incorporação dessa teoria de forma a modelar a não
estacionariedade intrínseca aos mercados é de interesse em diversas áreas de
finanças quantitativas, como otimização de portfólio~\cite{procacci}
~\cite{luo}~\cite{guastaroba} e precificação de opções~\cite{chorro}.
Outras áreas como gerenciamento de risco já são fundamentalmente baseadas em
modelagem uni e multivariada.


\subsubsection*{Referencial Teórico}

Demonstrar conhecimento da linha de pesquisa escolhida destacando em que
pontos a proposta de projeto poderá contribuir na expansão do estado da arte

\subsubsection*{Metodologia}

Demonstrar clareza em dar soluções para a linha de pesquisa escolhida

\subsubsection*{Cronograma}

Demonstrar exequibilidade da proposta, indicar possíveis disciplinas a cursar
e a organização do tempo durante seu período de vínculo ao curso

Disciplinas:

Núcleo Algoritmos: Projeto e Análise de Algoritmos, Programação Competitiva
Núcleo Estatística: FECD A, FECD B, Análise de Séries Temporais
Núcleo Otimização: Programação Linear, Programação Não Linear
Núcleo Específico: Finanças Quantitativas e Gerenciamento de Risco

1 semestre: PAA, FECD B
2 semestre: Machine Learning, Programação Competitiva
3 e 4: Programação Linear, Programação Não Linear, Análise de Séries Temporais,
Finanças Quantitativas (sujeito à oferta)

% \subsubsection*{Referências}

\bibliographystyle{ieeetr}
\bibliography{refs}

\end{document}
