\documentclass[a4paper, 12pt]{article}

\usepackage{geometry}
\usepackage{setspace}
\usepackage{titlesec}
\usepackage[utf8]{inputenc}
\usepackage[portuguese]{babel}
% \addtolength{\topsep}{-5mm}
% % \addtolength{\partopsep}{-1mm}
% \addtolength{\itemsep}{-5mm}
\usepackage{enumitem}
\setlist{nosep}

\begin{document}

\textbf{Modelagem Multivariada de Heteroscedasticidade para Estimativa de Risco
Dinâmico em Otimização de Portfólio}
\subsubsection*{Introdução}
\begin{spacing}{.9  }
O problema de otimização de portfólio é um desafio fundamental na teoria
financeira que envolve a alocação eficiente de recursos em diferentes ativos. O
objetivo é encontrar a combinação ideal de investimentos que maximize o retorno
esperado para um determinado nível de risco, levando em conta restrições de
investimento como limitações de alocação e liquidez. Central ao problema é a
definição de uma medida de risco, responsável por quantificar o potencial de
perda financeira associado à um portfólio%, sendo este um tópico de considerável
%atenção acadêmica~\cite{hoe2010}.

A medida de risco tradicional em otimização de portfólio é variância, seguindo
o método de Média-Variância (MV) proposto por Markowitz~\cite{markowitz}.
Apesar de conveniente devido à sua simplicidade esta medida não possui muitas
propriedades desejáveis~\cite{rachev}, tornando comum o uso de medidas mais
elaboradas~\cite{gambrah2014risk}. Um dos problemas da variância, compartilhado
por muitas medidas mais sofisticadas, é a
suposição que risco é estático no tempo, enquanto mercados financeiros são
dinâmicos e sujeitos à constante variação temporal~\cite{procacci}. Métricas de
risco que tentam capturar a natureza dinâmica de uma carteira de investimentos
são apropriadamente denominadas ``medidas de risco dinâmicas''.

A incorporação de medidas de risco dinâmicas em otimização de portfólio pode
ser útil para a geração de representações mais precisas de risco
instantâneo~\cite{chan1999portfolio} e para a construção de portfólios dinâmicos, cuja
distribuição de investimentos evolui ao longo do tempo por meio de políticas de
rebalanceamento~\cite{metin, holten}. Como várias medidas de risco são
calculadas a partir da matriz de covariância das séries temporais de preços ou
retornos dos ativos de uma carteira, a modelagem temporal da matriz de
covariância é uma forma de gerar métricas de risco dinâmicas que é bem estudada
na literatura~\cite{chan1999portfolio}.

A modelagem temporal da matriz de covariância de um conjunto de séries
temporais é essencialmente um problema de modelagem de heteroscedasticidade
multivariada. Dentre os modelos tradicionais com esta finalidade, os propostos
por Engle~\cite{bauwens} geram medidas de risco dinâmicas mais representativas
que suas contrapartidas estáticas~\cite{metin, holten, weirum}. Há uma série de
problemas no uso prático destes modelos que decorrem de sua natureza
paramétrica~\cite{morettin}, sugerindo o uso de modelos de \emph{machine
learning} (ML) como uma alternativa mais flexível.

A aplicação de algoritmos de ML para modelagem temporal de matrizes de
covariância atinge resultados melhores que abordagens tradicionais em alguns
trabalhos~\cite{svr, ann}. Aprendizado profundo, em particular, apresenta
resultados promissores em estudos recentes~\cite{multi_dl, dl2}.

O objetivo do projeto proposto é investigar soluções de otimização de portfólio
a partir de uma medida de risco dinâmica obtida por meio de modelagem
multivariada de heteroscedasticidade de preços e retornos de ativos
financeiros. Serão explorados modelos paramétricos, não paramétricos e híbridos
com uma ênfase em aprendizado profundo. (incluir isso?) Como objetivo
secundário será escrito um sistema para avaliação prática do método de
otimização de portfólio desenvolvido.

% É demonstrado na literatura que algoritmos de aprendizado de máquina não
% paramétricos são mais eficientes para alguns problemas de econometria
% financeira~\cite{hsu2016bridging}.


% O objetivo do projeto proposto é aprimorar soluções de otimização de portfólio
% a partir de uma medida de risco dinâmica obtida por meio de modelagem temporal
% multivariada de preços de instrumentos financeiros.

\subsubsection*{Referencial Teórico}

Um dos primeiros modelos para heteroscedasticidade de séries temporais foi
proposto por Bollerslev (1986) como uma continuação do trabalho de Engle (1982)
~\cite{garch}. O modelo de Bollerslev, cunhado \emph{Generalized Autoregressive
Conditional Heteroscedasticity} (GARCH), é amplamente usado em econometria
financeira desde sua criação. Tentativas iniciais de desenvolver um modelo
responsável pela representação de covariância condicional entre múltiplas
séries partiram de uma generalização do GARCH para o contexto multivariado,
resultando na classe denominada por Bauwens et. al~\cite{bauwens} como
\emph{Multivariate} GARCH (MGARCH).

É observado por múltiplos autores~\cite{bauwens, morettin} que o uso da maioria
dos modelos MGARCH possui uma série de dificuldades práticas quanto à
especificação de estrutura, identificação de modelo, ajuste de parâmetros e
complexidade computacional. Uma notável exceção é o \emph{Dynamic Conditional
Correlation} (DCC)~\cite{dcc}, desenvolvido com o propósito de contornar essas
limitações. Fizseder et. al~\cite{svr} argumenta que abordagens de aprendizado
de máquina não sofrem dos mesmo problemas devido ao fato de envolverem modelos
menos paramétricos. Christensen et. al~\cite{christensen} faz um estudo
comparativo da aplicação de diversos algoritmos de ML tradicionais para
previsão de variância condicional univariada, demonstrando superioridade em
relação à uma seleção de métodos econométricos. Xiong et. al~\cite{xiong}
realiza um estudo semelhante com redes neurais \emph{Long Short Term Memory}
(LSTM), concluindo expressiva melhora de desempenho da LSTM em relação a um
\emph{benchmark} GARCH. Fechar falando sobre ML e DL para multivariada?

Chan et. al (1999)~\cite{chan1999portfolio} apresenta um dos primeiros artigos
influentes a diretamente investigar o impacto do emprego de variância dinâmica,
obtida a partir de modelagem de covariância, no método de otimização MV. É
observado pelos autores que a variância dinâmica gera resultados melhores como
métrica de risco instantânea do que variância histórica. Modelos MGARCH são
utilizados para construção de portfólios com políticas de rebalanceamento
em~\cite{holten} e~\cite{weirum}, produzindo resultados melhores do que métodos
correspondentes baseados em risco histórico. Com o mesmo objetivo Metin
(2022)~\cite{metin} emprega o modelo DCC, motivado pela possibilidade de
inclusão de um maior número de ativos no portfólio dinâmico, obtendo resultados
positivos e mais flexíveis quando comparado ao uso de outros modelos MGARCH. Há
uma série de trabalhos com objetivo semelhante que utilizam de modelos de ML.
Redes neurais artificais são usadas em~\cite{ann}, produzindo resultados piores
que modelos econométricos elaborados, e regressão por vetor suporte é usado
em~\cite{svr}, atingindo acurácia e retornos melhores que um \emph{benchmark}
DCC. Em~\cite{dl2} a união de modelos LSTM e \emph{Convolutional Neural
Networks} (CNNs) é explorada a fim de aprimorar a extração de características
das séries modeladas, gerando resultados superiores à otimização baseada em
variância.

Boulet (2021)~\cite{dl_multi} afirma que apesar de promissoras, abordagens de
aprendizado profundo para previsão de matrizes de covariância foram pouco
exploradas. Neste mesmo trabalho é apresentado uma arquitetura híbrida que une
os modelos GARCH e LSTM, superando abordagens baseadas em MGARCH.

\subsubsection*{Metodologia}

p1: Detalhar o que será feito
p2: Descrever sistema desenvolvido para testar método
p3: Descrever dados utilizados
p4: Descrever métodos de avaliação


\subsubsection*{Cronograma}

Conforme estabelecido pela estrutura curricular do PPGCC será cursada a
disciplina Projeto e Análise de Algoritmos do Núcleo Comum. Das disciplinas das
linhas de pesquisa de Inteligência Artificial e Otimização há interesse do
candidato em cursar Fundamentos de Estatística para Ciência dos Dados B,
Aprendizado Profundo, Aprendizado de Máquina, Programação Não Linear e
Programação Estocástica. Das demais destaca-se a matéria Finanças Quantitativas
e Gerenciamento de Risco. As matérias serão cursadas de acordo com o
planejamento conjunto do aluno e orientador e oferta.

As atividades pertinentes à pesquisa do candidato serão distribuídas entre os
quatro semestres da seguinte maneira:
\begin{enumerate}
    \item Revisão de literatura e experimentos iniciais com base de dados M6.
    \item Início de desenvolvimento de sistema de otimização e elaboração do
    método proposto.
    \item Finalização do sistema de otimização. Experimentos com o método
    proposto. Ajustes no método.
    \item Avaliação de experimentos finais e redação de dissertação.
\end{enumerate}
\end{spacing}
\bibliographystyle{ieeetr}
\begingroup
\titleformat*{\section}{\fontsize{12pt}{12pt}\bfseries\selectfont}
\begin{spacing}{.7}
% {\footnotesize
\bibliography{refs}%}
\end{spacing}
\endgroup

\end{document}
