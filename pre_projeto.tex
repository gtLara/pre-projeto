\documentclass[a4paper, 12pt]{article}

\usepackage{geometry}
\usepackage{setspace}
\usepackage{titlesec}
\usepackage[utf8]{inputenc}
\usepackage[portuguese]{babel}
% \addtolength{\topsep}{-5mm}
% % \addtolength{\partopsep}{-1mm}
% \addtolength{\itemsep}{-5mm}
\usepackage{enumitem}
\setlist{nosep}

\begin{document}

\begin{center}
\textbf{Modelagem de heteroscedasticidade multivariada para estimativa de risco
dinâmico em otimização de portfólio}
\end{center}
\subsubsection*{Introdução}
\begin{spacing}{.9}

O problema de otimização de portfólio é um desafio fundamental na teoria
financeira que envolve a alocação eficiente de recursos em diferentes
ativos. Busca-se encontrar a combinação ideal de investimentos que maximize
o retorno esperado para um determinado nível de risco, considerando
restrições de investimento como limitações de alocação e liquidez. Central
ao problema é a definição de uma medida de risco, responsável por
quantificar o potencial de perda financeira associado a um portfólio.%,

A medida de risco tradicional em otimização de portfólio é variância, seguindo
o método de Média-Variância (MV) proposto por Markowitz~\cite{markowitz}.
Apesar de conveniente devido à sua simplicidade, essa medida possui
muitas propriedades indesejáveis~\cite{rachev}, tornando comum o uso de
medidas alternativas~\cite{gambrah2014risk}. Um dos problemas da
variância, compartilhado por muitas medidas mais sofisticadas, é a
suposição de que o risco é estático no tempo, enquanto mercados financeiros são
dinâmicos e sujeitos à constante variação temporal~\cite{procacci}.
Métricas de risco que tentam capturar a natureza dinâmica de uma carteira
de investimentos são apropriadamente denominadas ``medidas de risco
dinâmicas''.

A incorporação de medidas de risco dinâmicas em otimização de portfólio pode
ser útil para a geração de representações mais precisas de risco
instantâneo~\cite{chan1999portfolio}. Essas medidas são também úteis para a
construção de portfólios dinâmicos, cuja distribuição de investimentos
evolui ao longo do tempo por meio de políticas de
rebalanceamento~\cite{metin, holten}. Como várias medidas de risco são
calculadas a partir da matriz de covariância das séries temporais de preços
ou retornos dos ativos de uma carteira, a modelagem temporal da matriz de
covariância representa uma forma de gerar métricas de risco dinâmicas que é bem
estudada na literatura~\cite{chan1999portfolio}.

A modelagem temporal da matriz de covariância de um conjunto de séries
temporais é essencialmente um problema de modelagem de heteroscedasticidade
multivariada. Dentre os modelos tradicionais com tal finalidade, os propostos
por Engle~\cite{bauwens} geram medidas de risco dinâmicas mais representativas
que suas contrapartidas estáticas~\cite{metin, holten, weirum}. Há uma série de
problemas no uso prático destes modelos que decorrem de sua natureza
paramétrica~\cite{morettin}, sugerindo o uso de modelos de \emph{machine
learning} (ML) como uma alternativa mais flexível.

A aplicação de algoritmos de ML para modelagem temporal de matrizes de
covariância atinge resultados melhores que abordagens tradicionais em
alguns trabalhos~\cite{svr, ann}. A abordagem de aprendizado profundo, em
particular, apresenta resultados promissores em estudos
recentes~\cite{dl_multi, dl2, fang2021cnn}.

Dessa forma, o objetivo do projeto proposto é investigar soluções de otimização
de portfólio a partir de uma medida de risco dinâmica obtida por meio de
modelagem multivariada de heteroscedasticidade de preços e retornos de
ativos financeiros. Serão explorados modelos paramétricos, não paramétricos
e híbridos com uma ênfase em aprendizado profundo.

\subsubsection*{Referencial Teórico}

Um dos primeiros modelos para heteroscedasticidade de séries temporais
univariadas foi proposto por Bollerslev (1986) como uma continuação do trabalho
de Engle (1982) ~\cite{garch}. O modelo de Bollerslev, cunhado
\emph{Generalized Autoregressive Conditional Heteroscedasticity} (GARCH), é
amplamente usado em econometria financeira desde sua criação. Tentativas
iniciais de desenvolver um modelo responsável pela representação de covariância
condicional entre múltiplas séries partiram de uma generalização do GARCH para
o contexto multivariado, resultando na classe denominada por Bauwens et.
al~\cite{bauwens} como \emph{Multivariate} GARCH (MGARCH).

É observado por múltiplos autores~\cite{bauwens, morettin} que o uso da maioria
dos modelos MGARCH possui uma série de dificuldades práticas quanto à
especificação de estrutura, identificação de modelo, ajuste de parâmetros e
complexidade computacional. Uma notável exceção é o \emph{Dynamic Conditional
Correlation} (DCC)~\cite{dcc}, desenvolvido com o propósito de contornar essas
limitações. Fizseder et. al~\cite{svr} argumentam que abordagens de aprendizado
de máquina não sofrem dos mesmos problemas devido ao fato de envolverem menos
suposições sobre as distribuições modeladas. Christensen et. al~\cite{christensen} faz um estudo
comparativo da aplicação de diversos algoritmos de ML tradicionais para
previsão de variância condicional univariada, demonstrando superioridade em
relação a uma seleção de métodos econométricos. Xiong et. al~\cite{xiong}
realizam um estudo semelhante com redes neurais \emph{Long Short Term Memory}
(LSTM)~\cite{LSTM}, concluindo expressiva melhora de desempenho da LSTM em relação a um
\emph{benchmark} GARCH. Aprendizado profundo é utilizado para previsão de
matrizes de covariância em~\cite{fang2021cnn}, em que os autores usam
\emph{Convolutional} LSTM (ConvLSTM)~\cite{convlstm} obtendo acurácia superior
à de múltiplos modelos tradicionais.


Chan et. al (1999)~\cite{chan1999portfolio} apresenta um dos primeiros artigos
influentes a diretamente investigar o impacto do emprego de variância dinâmica,
obtida a partir de modelagem de covariância, no método de otimização MV. É
observado pelos autores que a variância dinâmica gera resultados melhores como
métrica de risco instantânea do que variância histórica. Modelos MGARCH são
utilizados para construção de portfólios com políticas de rebalanceamento
em~\cite{holten} e~\cite{weirum}, produzindo resultados melhores do que métodos
correspondentes baseados em risco histórico. Com o mesmo objetivo, Metin
(2022)~\cite{metin} emprega o modelo DCC, motivado pela possibilidade de
inclusão de um maior número de ativos no portfólio dinâmico, obtendo resultados
positivos e mais flexíveis quando comparado ao uso de outros modelos MGARCH. Há
uma série de trabalhos com propósito semelhante que utilizam modelos de ML.
Redes neurais artificais são usadas em~\cite{ann}, produzindo resultados piores
que os de modelos econométricos elaborados, e regressão por vetor suporte é
usada em~\cite{svr}, atingindo acurácia e retornos melhores que um
\emph{benchmark} DCC. Em~\cite{dl2} a união de modelos LSTM e
\emph{Convolutional Neural Networks} (CNNs)~\cite{cnn} é explorada a fim de
aprimorar a extração de características das séries modeladas, gerando
resultados superiores à otimização baseada em variância tradicional.

Boulet (2021)~\cite{dl_multi} afirma que, apesar de promissoras, abordagens de
aprendizado profundo para previsão de matrizes de covariância foram pouco
exploradas. Nesse mesmo trabalho é apresentada uma arquitetura híbrida que une
os modelos GARCH e LSTM, superando abordagens baseadas em MGARCH.

Observa-se que, embora emergentes, métodos de ML estão próximos ao estado da
arte. O contexto estabelecido anteriormente motiva o estudo de abordagens de
aprendizado de máquina ao problema discutido, com o desenvolvimento de
arquiteturas neurais profundas híbridas entre modelos MGARCH, CNNs e LSTMs
sendo de particular interesse.

\subsubsection*{Metodologia}

Inicialmente serão desenvolvidas duas soluções base cujo desempenho será usado
para comparativamente avaliar o método proposto. Essas soluções serão compostas
por um otimizador MV com variância histórica e outro com variância dinâmica
calculada a partir do modelo DCC. O portfólio otimizado pela solução dinâmica
será rebalanceado diariamente, como em~\cite{metin}. Os dados usados para
validar os \emph{benchmarks} em um primeiro momento serão as séries temporais
da competição M6~\cite{m6}, composta de preços de cinquenta ações do índice
S\&P500 e \emph{exchange traded funds} (ETFs).

Em seguida, após uma extensa revisão de literatura, serão conduzidos
experimentos sob condições idênticas com os modelos econométricos e de ML mais
promissores. Será explorado o uso de redes LSTM e ConvLSTM com mecanismos de
atenção~\cite{attention} e CNNs. As combinações desses modelos serão
investigadas de forma a averiguar o potencial de um modelo híbrido, composto
por DCC e ConvLSTM baseada em atenção, na expansão do estado da arte, seguindo
os resultados positivos de~\cite{dl_multi}. Este modelo híbrido é a principal
contribuição sugerida pelo projeto. Após a validação e prototipagem do modelo
desenvolvido, ainda utilizando a base de dados da competição M6, será explorado
o impacto de medidas de risco dinâmicas alternativas, como Valor em Risco (VaR)
e \emph{limited expected loss}~\cite{gambrah2014risk}, calculadas a partir das
matrizes de covariância históricas e previstas pelos métodos base e proposto.
Durante os experimentos desta fase, a política de rebalanceamento será variada
apenas quanto ao número de dias entre rebalanceamentos consecutivos.

Com o protótipo do modelo híbrido desenvolvido e validado, métricas de risco
implementadas e intervalo de rebalanceamento definido, o projeto irá alterar seu
foco para validação da proposta em uma diversidade maior de ativos financeiros
reais. Será desenvolvido um sistema de investimentos a partir de APIs
financeiras disponíveis para testar o método sugerido em tempo real e ativos
mais diversificados como \emph{commodities} e cryptomoedas.

Por fim, serão realizadas avaliações da acurácia do modelo proposto em prever
matrizes de covariâncias e, de forma mais geral, dos retornos obtidos pelo
portfólio dinâmico sob as medidas de risco exploradas. A linguagem de
programação utilizada no projeto será \verb+Python+ e o \emph{framework} de
aprendizado profundo \verb+PyTorch+.

\subsubsection*{Cronograma}

Conforme sugerido pela estrutura curricular do PPGCC, serão cursadas as
matérias Projeto e Análise de Algoritmos e Fundamentos Teóricos da Computação
do núcleo comum. Das disciplinas da linha de pesquisa em Inteligência
Artificial e Robótica, há interesse do candidato em cursar Inteligência
Artificial e Aprendizado de Máquina. Das demais, destaca-se a
matéria Programação Não Linear. Além disso, há obrigatoriedade do curso de Estágio
em Docência. Essas matérias, cuja carga horária soma a vinte e dois créditos,
serão cursadas de acordo com sua oferta e planejamento em conjunto entre o
candidato e seu orientador.

As atividades pertinentes à pesquisa do candidato serão distribuídas entre os
quatro semestres da seguinte maneira:
\begin{enumerate}
    \item Revisão de literatura e experimentos iniciais com a base de dados M6.
    \item Início de desenvolvimento de sistema de otimização e elaboração do
    método proposto.
    \item Finalização do sistema de otimização. Experimentos com o método
    proposto. Ajustes no método.
    \item Avaliação de experimentos finais e redação de dissertação.
\end{enumerate}
\end{spacing}
\bibliographystyle{ieeetr}
\begingroup
\titleformat*{\section}{\fontsize{12pt}{12pt}\bfseries\selectfont}
\begin{spacing}{.5}
% {\footnotesize
\bibliography{refs}%}
\end{spacing}
\endgroup

\end{document}
